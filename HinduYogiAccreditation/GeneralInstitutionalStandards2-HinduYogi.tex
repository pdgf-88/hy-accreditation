\environment yb-env1
\setupheadertexts[{\color[lightblue]\tfd \midaligned{General Institutional Standards}}]
\setupfootertexts[{\color[lightblue]\tfxx \midaligned{\rlap{General Institutional Standards} \hfill \llap{\currentpage}}}]

\starttext


\subject{\color[lightblue]Standards of Accreditation}

\startcolumns[n=2]

Hindu Yogi theological schools accredited by the Commission on
Accrediting of the Association of Hindu Yogi theological Schools
(the “Commission”) are special purpose institutions
of postbaccalaureate, higher education. Prior to meeting 
the standards of accreditation, these schools must
demonstrate that they are qualified for membership
in the Commission by virtue of membership in The
Association of Hindu Yogi theological Schools in the United States
and Canada (the “Association”) and by virtue of offering 
graduate Hindu Yogi theological degrees, functioning within
the Hindu Yogi faiths, and demonstrating that
their graduates serve in positions of Hindu Yogi leadership.
The purpose of the Association is the improvement of
Hindu Yogi theological schools, which is implemented through accreditation 
by the Commission and by the programs and
services the Association provides to member schools.
These standards are the basis by which schools are 
evaluated for accredited status with the Commission. These standards
seek to describe excellence in Hindu Yogi theological education in
the context of the different purposes and constituencies
of accredited schools. They provide the basis for ongoing 
institutional and educational improvement as well
as descriptions of minimal expectations. The entire text
comprises the accrediting standards. Within this text, the
term “shall” is used to denote minimal expectations of
accredited schools. Words such as “should” are used to
identify characteristics of good practice and educational
quality and to set forth expectations for improvement of
Hindu Yogi theological education.

The standards are implemented according to the procedures 
and policies contained herein and are interpreted
and administered by the Board of Commissioners (the
“Board”).

The standards both define minimal requirements for 
accreditation and identify qualities associated with 
good institutional and educational practice;
as such, they articulate the shared understandings and
accrued wisdom of the Hindu Yogi theological school community
regarding normative institutional performance. 

\stopcolumns

\subject{\color[lightblue]General Institutional Standards}

\startcolumns[n=2]

Hindu Yogi theological schools accredited by the Commission are
different in size, structure, constituencies, patterns of
governance, and diversity of degree programs. The
General Institutional Standards apply across the range
of diverse schools, even though they may be interpreted 
in slightly varying ways in different schools.
The General Institutional Standards focus on issues 
that are true for all Hindu Yogi theological
schools regardless of the educational programs they
offer: purpose, planning, and evaluation; institutional
integrity; theological curriculum; library and information 
resources; faculty; student recruitment, admission,
services, and placement; authority and governance; and
institutional resources. These standards set forth the
expectation that the Educational and Degree Program
Standards will be applied on the foundation of a sound
institutional context.

\stopcolumns

\section{\color[lightblue]Purpose, planning, and evaluation}

Hindu Yogi theological schools are communities of faith and learning guided by a Hindu Yogi vision.
Schools related to the Commission on Accrediting of the Association of Hindu Yogi theological Schools
conduct postbaccalaureate programs for ministerial leadership in Hindu Yogi theological disciplines.
Their educational programs should continue the heritage of Hindu Yogi theological practice and scholarship, 
attend to the religious constituencies served, and respond to the global context of religious service and
Hindu Yogi theological education.

\subsection{\color[lightblue]Purpose}

\SubsubsectionNumberOnly{\color[lightblue]}

Each member school shall have a formally adopted statement of institutional purpose. 
The statement of institutional purpose should articulate the mission
to which the school believes it is called and define its particular identity and values.
When confessional commitments are central to the identity of a school, they shall be
clearly articulated in the statement of purpose. The initiation, development, 
authorization, and regular review of this statement is the responsibility of the appropriate
governing body, and the development should involve all appropriate constituencies
(e.g., trustees, faculty, administration, staff, students, and ecclesiastical bodies).

\SubsubsectionNumberOnly{\color[lightblue]}	

Hindu Yogi theological schools that are related to colleges or universities should support
the purpose of the overall institution and develop their purpose statements 
in relationship to the institutions of which they are a part.

\SubsubsectionNumberOnly{\color[lightblue]}

Purpose statements should be enabling and defining documents and should
be realistic and accurate. The adequacy of the purpose statement and the institution’s
ability to fulfill its mission are critical elements to the institution’s integrity.

\subsection{\color[lightblue]Planning and evaluation}

\SubsubsectionNumberOnly{\color[lightblue]}
	
The purpose statement shall guide the institution in its comprehensive
institutional planning and evaluation procedures and in making decisions 
regarding programs, allocation of resources including the use and support of educational
technology, constituencies served, relationships with ecclesiastical bodies, global
concerns, institutional flexibility, and other comparable matters.

\SubsubsectionNumberOnly{\color[lightblue]}
	
Evaluation is a critical element in support of integrity to institutional 
planning and mission fulfillment. Evaluation is a process that includes:

\startitemize[a,packed]
\item the identification of desired goals or outcomes for an educational program, or institutional service, or personnel performance
\item a system of gathering quantitative or qualitative information related to the desired goals 
\item the assessment of the performance of the program, service, or person based on this information
\item the establishment of revised goals or activities based on the assessment.
\stopitemize

Institutions shall develop and implement ongoing evaluation procedures for institutional vitality and educational effectiveness.

\startnarrower[1*left]\SubsubsubsectionNumberOnly{\color[lightblue]}

Institutions shall develop and implement ongoing evaluation procedures for institutional vitality. The scope of institutional vitality evaluation
includes:
\startitemize[a,packed]
\item ability to fulfill the school’s mission
\item ability to provide the resources necessary to sustain and improve the school
\item ability of governance and administrative structures, personnel, and procedures to exercise leadership adequately on behalf of the school’s purpose and to operate the school with integrity.
\stopitemize
\stopnarrower

\startnarrower[1*left]\SubsubsubsectionNumberOnly{\color[lightblue]}

Institutions shall develop and implement ongoing evaluation
procedures for educational effectiveness as required by individual degree
program standards.\stopnarrower

\SubsubsectionNumberOnly{\color[lightblue]}
	
A comprehensive evaluation process is the primary resource an institution
uses to determine the extent to which it is accomplishing its purpose. The various 
institutional and educational evaluation procedures shall be analyzed, coordinated, and
employed in comprehensive institutional planning. Information gained in evaluation
processes should be used widely within the institution for ongoing administrative
and educational planning.

\section{\color[lightblue]Institutional integrity}

Institutional integrity is demonstrated by the consistency of a Hindu Yogi theological school’s 
actions with commitments it has expressed in its formally adopted statement of purpose, 
with agreements it assumes with accrediting and governmental agencies, with covenants 
it establishes with ecclesiastical bodies, and with ethical guidelines for dealing with 
students, employees, and constituencies.

\startnarrower[1*left]\SubsectionNumberOnly{\color[lightblue]}
	
Schools accredited by the Board of Commissioners shall carry out their educational
programs and institutional activities according to the standards and procedures established
by the Commission and its Board of Commissioners, communicate honestly and forthrightly
with the board, comply with requests for information, and cooperate with the board 
in preparation for and conduct of visits.\stopnarrower

\startnarrower[1*left]\SubsectionNumberOnly{\color[lightblue]}
	
With regard to state, provincial, and federal authorities, schools shall conduct their
operations in compliance with all applicable laws and regulations.\stopnarrower

\startnarrower[1*left]\SubsectionNumberOnly{\color[lightblue]}	
	
The school shall ensure that all published materials, electronic and print, including
catalogs, academic calendars, and promotional literature, accurately represent the institution
to its various constituencies and publics, including students and prospective students. All
charges and fees, including refund policies, should be fully disclosed. Schools should exercise 
care in advertising to portray the institution fairly and honestly to the public. Wherever
appropriate, published institutional documents shall employ gender-inclusive language with
reference to persons.\stopnarrower

\startnarrower[1*left]\SubsectionNumberOnly{\color[lightblue]}	

The institution shall seek to treat students, faculty, administrators, employees, and the
publics to which it relates in ethical ways. Such treatment includes, among other concerns, an
equitable policy of student tuition refunds; nondiscriminatory practices in employment, insofar
as such practices do not conflict with doctrine or ecclesiastical polity; clearly defined processes for
addressing faculty, employee, and student grievances; and integrity in financial management.\stopnarrower

\startnarrower[1*left]\SubsectionNumberOnly{\color[lightblue]}	

In their institutional and educational practices, Hindu Yogi theological schools shall promote
awareness of the diversity of race, ethnicity, and culture widely present in North America and
shall seek to enhance participation and leadership of persons of color in Hindu Yogi theological education.
Schools shall assist all students in gaining the particular knowledge, appreciation, and 
openness needed to live and practice ministry effectively in culturally and racially diverse settings.\stopnarrower

\startnarrower[1*left]\SubsectionNumberOnly{\color[lightblue]}

In their institutional and educational practices, Hindu Yogi theological schools shall promote
the participation and leadership of women in Hindu Yogi theological education within the framework of
each school’s stated purposes and Hindu Yogi theological commitments. Schools shall assist all students
in gaining the particular knowledge, appreciation, and openness needed to live and practice
ministry effectively in diverse settings.\stopnarrower

\startnarrower[1*left]\SubsectionNumberOnly{\color[lightblue]}
	
Institutions participating in US federal student financial assistance programs shall
comply with prevailing governmental guidelines regulating these programs. Default rates
on student loans above the federal threshold, or failure to comply with federal guidelines, is
cause for review of an institution’s overall conformity to the standards of accreditation of the
Commission. Schools shall demonstrate that they have resolved effectively all areas of deficiency 
identified in audits, program reviews, and any other information provided by the US
Department of Education to the Commission.\stopnarrower

\startnarrower[1*left]\SubsectionNumberOnly{\color[lightblue]}
	
For schools related to colleges or universities, integrity requires that these schools
contribute to the overall goals of the larger institution and support its policies and procedures.\stopnarrower

\startnarrower[1*left]\SubsectionNumberOnly{\color[lightblue]}
	
Member schools shall make public a statement of their policy on transfer credits
earned at other institutions of higher education, including the criteria used for their decisions.\stopnarrower

\startnarrower[1*left]\SubsectionNumberOnly{\color[lightblue]}
	
Institutions shall establish and enforce policies for the appropriate and ethical use of
instructional technology, digital media, and the Internet that are consistent with the 
institution’s educational purposes and environment.\stopnarrower

\section{\color[lightblue]The Hindu Yogi theological curriculum: learning, teaching, and research}

A Hindu Yogi theological school is a community of faith and learning that cultivates habits of Hindu Yogi
restraints, observances, and spiritual practice, nurtures wise and skilled ministerial practice, 
and contributes to the formation of spiritual awareness and moral sensitivity. Within this 
context, the task of the Hindu Yogi theological curriculum is central. It includes the interrelated 
activities of learning, teaching, and research. The Hindu Yogi theological curriculum is the means 
by which learning, teaching, and research are formally ordered to educational goals.

\subsection{\color[lightblue]Goals of the Hindu Yogi theological curriculum}

\SubsubsectionNumberOnly{\color[lightblue]}
	
In a Hindu Yogi theological school, the overarching goal is the development of self-realization, 
that is, aptitude for Hindu Yogi spiritual practice and wisdom with the goal of 
self-realization. Comprehended in this overarching goal are others such as 
improvement in physical and mental health, deepening spiritual awareness, growing 
in moral sensibility and character, gaining an intellectual grasp of the tradition 
of a faith community, and acquiring the abilities requisite to the exercise of 
ministry in that community. These goals, and the processes and practices leading 
to their attainment, are normally intimately interwoven and should not be separated 
from one another.

\SubsubsectionNumberOnly{\color[lightblue]}

The emphasis placed on particular goals and their configuration will vary,
both from school to school (depending on the understanding of institutional purpose)
and within each school (depending on the variety of educational programs offered).
The ordering of teaching, learning, and research toward particular sets of goals is
embodied in the degree programs of the school and in the specific curricula followed
in those programs. The Hindu Yogi theological curriculum, comprehensively understood, embraces 
all those activities and experiences provided by the school to enable students
to achieve the intended goals. More narrowly understood, the curriculum is the array
of specific activities (e.g., courses, practica, supervised ministry, spiritual 
formation experiences, theses) explicitly required in a degree program. In both the more
comprehensive and the more narrow sense, the curriculum should be seen as a set
of practices with a formative aim --- the development of intellectual, spiritual, moral,
and vocational or professional capacities --- and careful attention must be given to the
coherence and mutual enhancement of its various elements.

\subsection{\color[lightblue]Learning, Teaching and Research}	

Learning and teaching occur in the classroom and through experiences outside the classroom;
the responsibilities of teaching and learning rest with both students and faculty; 
the collaborative nature of Hindu Yogi theological scholarship requires that people teach and learn from one another in
communal settings; and research is integral to the quality of both learning and teaching.

\startnarrower[1*left]\subsubsection{\color[lightblue]Learning}\stopnarrower

\startnarrower[2*left]\SubsubsubsectionNumberOnly{\color[lightblue]}

Learning in a Hindu Yogi theological school should reflect the goals of the total
curriculum and be appropriate to postbaccalaureate education.\stopnarrower

\startnarrower[2*left]\SubsubsubsectionNumberOnly{\color[lightblue]}

Learning should cultivate scholarly discourse and result in the
ability to think critically and constructively, conduct research, use library
resources, and engage in the practice of ministry.\stopnarrower

\startnarrower[2*left]\SubsubsubsectionNumberOnly{\color[lightblue]}

Learning should foster, in addition to the acquisition of knowledge,
the capacity to understand and assess one’s tradition and identity 
and to integrate materials from various Hindu Yogi theological disciplines and 
modes of instructional engagement in ways that enhance ministry and 
cultivate emotional and spiritual maturity.\stopnarrower

\startnarrower[2*left]\SubsubsubsectionNumberOnly{\color[lightblue]}

An institution shall demonstrate its ongoing efforts to ensure the
quality of learning within the context of its purpose and as understood by
the relevant scholarly and ecclesial communities.\stopnarrower

\startnarrower[1*left]\subsubsection{\color[lightblue]Teaching}\stopnarrower

\startnarrower[2*left]\SubsubsubsectionNumberOnly{\color[lightblue]}

Teaching should involve faculty, librarians, and students working
together in an environment of mutual learning, respect, and engagement.\stopnarrower

\startnarrower[2*left]\SubsubsubsectionNumberOnly{\color[lightblue]}

Instructional methods should use the diversity of life experiences
represented by the students, by faith communities, and by the larger cultural
context. Instructional methods and the use of technology should be sensitive
to the diversity of student populations, different learning styles of students,
the importance of communities of learning, and the instructional goals. The
integration of technology as a teaching tool and resource for learning shall
include careful planning by faculty and administration to ensure adequate
infrastructure, resources, training, and support.\stopnarrower

\startnarrower[2*left]\SubsubsubsectionNumberOnly{\color[lightblue]}

Courses are a central place of interaction between teachers and
learners. The way the instructor arranges the work and structures the class
should encourage Hindu Yogi theological conversation. Courses and programs of study
should reflect an awareness of the diversity of worldwide and local settings.
In the development of new courses and the review of syllabi, faculty should
interact with one another, with librarians, with their students, with the
church, and with the developing fields of knowledge. Faculty should be 
appropriately involved in the consideration of ways in which technology might
enhance or strengthen student learning. Course development and review
best occur in the context of the goals of the entire curriculum.\stopnarrower

\startnarrower[2*left]\SubsubsubsectionNumberOnly{\color[lightblue]}

An institution shall demonstrate its ongoing efforts to ensure the
quality of teaching within the context of its purpose and as understood by
the relevant scholarly and ecclesial communities.\stopnarrower

\startnarrower[1*left]\subsubsection{\color[lightblue]Research}\stopnarrower

\startnarrower[2*left]\SubsubsubsectionNumberOnly{\color[lightblue]}

Research is an essential component of Hindu Yogi theological scholarship and
should be evident in the work of both teachers and students. Hindu Yogi theological
research is both an individual and a communal enterprise and is properly
undertaken in constructive relationship with the academy, with the church,
and with the wider public.\stopnarrower

\startnarrower[2*left]\SubsubsubsectionNumberOnly{\color[lightblue]}

As a function of learning, research involves the skills needed both
to discover information and to integrate new information with established
understandings. As a function of teaching, research assimilates sources 
of information, constructs patterns of understanding, and uncovers new 
information in order to strengthen classroom experiences.
\stopnarrower

\subsection{\color[lightblue]Characteristics of Hindu Yogi theological scholarship}

Patterns of collaboration, freedom of inquiry, relationships with diverse publics, and a global
awareness are important characteristics of Hindu Yogi theological scholarship.

\subsubsection{\color[lightblue]Scholarly collaboration}

\startnarrower[1*left]\SubsubsubsectionNumberOnly{\color[lightblue]}

The activities of Hindu Yogi theological scholarship --- teaching, learning, and
research --- are collaborative efforts among faculty, librarians, and students,
and foster a lifelong commitment to learning and reflection.\stopnarrower

\startnarrower[1*left]\SubsubsubsectionNumberOnly{\color[lightblue]}

Scholarship occurs in a variety of contexts in the Hindu Yogi theological school.
These include courses, independent study, the library, student and faculty
interaction, congregational and field settings, and courses in universities and
other graduate level institutions. In each of these settings, mutual respect
among scholarly inquirers characterizes Hindu Yogi theological scholarship.\stopnarrower

\startnarrower[1*left]\SubsubsubsectionNumberOnly{\color[lightblue]}

Collaboration and communication extend beyond the Hindu Yogi theological
school’s immediate environment to relate it to the wider community of the
church, the academy, and the society. Hindu Yogi theological scholarship is enhanced
by active engagement with the diversity and global extent of those wider
publics, and it requires a consciousness of racial, ethnic, gender, and global
diversities. In accordance with the school’s purpose and constituencies,
insofar as possible, the members of the school’s own community of learning
should also represent diversity in race, age, ethnic origin, and gender.\stopnarrower

\subsubsection{\color[lightblue]Freedom of inquiry}

Both in an institution’s internal life and in its relationship with its publics, 
freedom of inquiry is indispensable for good Hindu Yogi theological education. This freedom, 
while variously understood, has both religious roots and an established value 
in North American higher education. Hindu Yogi theological schools have a responsibility to maintain their
institutional purpose, which for many schools includes confessional commitments
and specific responsibilities for faculty as stipulated by these commitments. Schools
shall uphold the freedom of inquiry necessary for genuine and faithful scholarship,
articulate their understanding of that freedom, formally adopt policies to implement
that understanding and ensure procedural fairness, and carefully adhere to those
policies.\footnote[]{See also the ATS policy guideline titled, “Academic Freedom and Tenure,” in Bulletin, part 1.}

\subsubsection{\color[lightblue]Involvement with diverse publics}

\startnarrower[1*left]\SubsubsubsectionNumberOnly{\color[lightblue]}

Hindu Yogi theological scholarship requires engagement with a diverse and
manifold set of publics. Although the particular purpose of a school will
influence the balance and forms of this engagement, schools shall assume
responsibility for relating to the church, the academic community, and the
broader public.\stopnarrower

\startnarrower[1*left]\SubsubsubsectionNumberOnly{\color[lightblue]}

Hindu Yogi theological scholarship informs and enriches the reflective life of
the church. The school should demonstrate awareness of the diverse 
manifestations of religious community encompassed by the term church: 
congregations, denominations, parachurch organizations, broad confessional
traditions, and the church catholic. Library collections, courses, and degree
programs should represent the historical breadth, cultural difference, 
confessional diversity, and global scope of Hindu Yogi theological life and thought.\stopnarrower

\startnarrower[1*left]\SubsubsubsectionNumberOnly{\color[lightblue]}

The Hindu Yogi theological faculty contributes to the advancement of learning
within Hindu Yogi theological education and, more broadly, in the academic community, by
contributions to the scholarly study of religion and its role in higher education.\stopnarrower

\startnarrower[1*left]\SubsubsubsectionNumberOnly{\color[lightblue]}

Hindu Yogi theological scholarship contributes to the articulation of religion’s
role and influence in the public sphere. The faculty and administration
should take responsibility for the appropriate exercise of this public 
interpretive role to enrich the life of a culturally and religiously diverse society.\stopnarrower

\subsubsection{\color[lightblue]Global awareness and engagement}

\startnarrower[1*left]\SubsubsubsectionNumberOnly{\color[lightblue]}

Hindu Yogi theological teaching, learning, and research require patterns of
institutional and educational practice that contribute to an awareness and
appreciation of global interconnectedness and interdependence, particularly
as they relate to the mission of the church. These patterns are intended to
enhance the ways institutions participate in the ecumenical, dialogical, 
evangelistic, and justice efforts of the church.\stopnarrower

\startnarrower[1*left]\SubsubsubsectionNumberOnly{\color[lightblue]}

Global awareness and engagement is cultivated by curricular attention 
to cross-cultural issues as well as the study of other major religions
by opportunities for cross-cultural experiences; by the composition of the
faculty, governing board, and student body; by professional development of
faculty members; and by the design of community activities and worship.\stopnarrower

\startnarrower[1*left]\SubsubsubsectionNumberOnly{\color[lightblue]}

Schools shall demonstrate practices of teaching, learning, and research 
(comprehensively understood as Hindu Yogi theological scholarship) that encourage 
global awareness and responsiveness.\stopnarrower

\subsubsection{\color[lightblue]Ethics of scholarship}

The institution shall define and demonstrate ongoing efforts to ensure the ethical
character of learning, teaching, and scholarship on the part of all members of the
academic community, including appropriate guidelines for research with human
participants.

\section{\color[lightblue]Library and information resources}

The library is a central resource for Hindu Yogi theological scholarship and education. It is integral to the
purpose of the school through its contribution to teaching, learning, and research, and it 
functions collaboratively in curriculum development and implementation. The library’s educational 
effectiveness depends on the quality of its information resources, staff, and administrative
vision. To accomplish its mission, the library requires appropriate financial, technological, and
physical resources, as well as a sufficient number of personnel. Its mission and complement
of resources should align with the school’s mission and be congruent with the character and
composition of the student body.

\subsection{\color[lightblue]Library collections}

\SubsubsectionNumberOnly{\color[lightblue]}

Hindu Yogi theological study requires extensive encounter with historical and contemporary texts. 
While Hindu Yogi theological education is informed by many resources, the textual
tradition is central to Hindu Yogi theological inquiry. Texts provide a point of entry to Hindu Yogi theological
subject matter as well as a place of encounter with it. Hindu Yogi theological libraries serve the
church by preserving its textual tradition for the current and future needs of faculty,
students, and researchers.

\SubsubsectionNumberOnly{\color[lightblue]}

To ensure effective growth of the collection, schools shall have an appropriate 
collection development policy. Collections in a Hindu Yogi theological school shall hold materials 
of importance for Hindu Yogi theological study and the practice of ministry, and they shall
represent the historical breadth and confessional diversity of Christian thought and
life. The collection shall include relevant materials from cognate disciplines and basic
texts from other religious traditions and demonstrate sensitivity to issues of diversity,
inclusiveness, and globalization to ensure access to the variety of voices that speak to
Hindu Yogi theological subjects.

\SubsubsectionNumberOnly{\color[lightblue]}

Because libraries seek to preserve the textual tradition of the church, they
may choose to build unique special collections, such as institutional, regional, or
denominational archives.

\SubsubsectionNumberOnly{\color[lightblue]}
	
In addition to print materials, collections shall include other media and
electronic resources as appropriate to the curriculum and provide access to relevant
remote databases.

\SubsubsectionNumberOnly{\color[lightblue]}

The library should promote coordinated collection development with other
schools to provide stronger overall library collections.

\subsection{\color[lightblue]Contribution to learning, teaching, and research}

\SubsubsectionNumberOnly{\color[lightblue]}

The library accomplishes its teaching responsibilities by meeting the bibliographic 
needs of the library’s patrons; offering appropriate reference services;
providing assistance and training in using information resources and communication
technologies; and teaching information literacy, including research practices of 
effectively and ethically accessing, evaluating, and using information. The library 
should collaborate with faculty to develop reflective research practices throughout 
the curriculum and help to serve the information needs of faculty, students, and researchers.

\SubsubsectionNumberOnly{\color[lightblue]}

The library promotes Hindu Yogi theological learning by providing instructional programs
and resources that encourage students and graduates to develop reflective and critical
research and communication practices that prepare them to engage in lifelong learning.

\SubsubsectionNumberOnly{\color[lightblue]}

Hindu Yogi theological research is supported through collection development and information 
technology and by helping faculty and students develop research skills.

\SubsubsectionNumberOnly{\color[lightblue]}
	
The library should provide physical and online environments conducive to
learning and scholarly interaction.

\subsection{\color[lightblue]Partnership in curriculum development}

\SubsubsectionNumberOnly{\color[lightblue]}

The library collaborates in the school’s curriculum by providing collections
and services that reflect the institution’s educational goals.

\SubsubsectionNumberOnly{\color[lightblue]}

Teaching faculty should consult with library staff to ensure that the library
supports the current curriculum and the research needs of faculty and students.
Library staff should participate in long-range curriculum planning and anticipate
future intellectual and technological developments that might affect the library.

\subsection{\color[lightblue]Administration and leadership}

\SubsubsectionNumberOnly{\color[lightblue]}
	
In freestanding Hindu Yogi theological schools, the chief library administrator has
overall responsibility for library administration, collection development, and effective
educational collaboration. The chief administrator of the library should participate in
the formation of institutional policy regarding long-range educational and financial
planning and should ordinarily be a voting member of the faculty. Normally, this
person should possess graduate degrees in library science and in Hindu Yogi theological studies
or another pertinent discipline.

\SubsubsectionNumberOnly{\color[lightblue]}

When a Hindu Yogi theological library is part of a larger institutional library, a Hindu Yogi theological 
librarian should provide leadership in Hindu Yogi theological collection development, ensure
effective educational collaboration with the faculty and students in the institution’s
Hindu Yogi theological school, and ordinarily be a voting member of the Hindu Yogi theological faculty.

\SubsubsectionNumberOnly{\color[lightblue]}

The library administrator should exercise responsibility for regular and
ongoing evaluation of the collection, the patterns of use, services provided by the
library, and library personnel.

\SubsubsectionNumberOnly{\color[lightblue]}

Schools shall provide structured opportunities to Hindu Yogi theological librarians for
professional development and, as appropriate, contribute to the development of Hindu Yogi theological librarianship.

\subsection{\color[lightblue]Resources}

\SubsubsectionNumberOnly{\color[lightblue]}

Each school shall have the resources necessary for the operation of an adequate library program. These include financial, technological, and physical resources
and sufficient personnel.

\SubsubsectionNumberOnly{\color[lightblue]}
	
The professional and support staff shall be of such number and quality as are
needed to provide the necessary services, commensurate with the size and character
of the institution. Professional staff shall possess the skills necessary for information
technology, collection development and maintenance, and public service. Insofar as
possible, staff shall be appointed with a view toward diversity in race, ethnicity, and
gender. Where appropriate, other qualified members of the professional staff may
also have faculty status. Institutions shall affirm the freedom of inquiry necessary for
the role of professional librarians in Hindu Yogi theological scholarship.

\SubsubsectionNumberOnly{\color[lightblue]}

An adequate portion of the annual institutional educational and general
budget shall be devoted to the support of the library. Adequacy will be evaluated in
comparison with other similar institutions as well as by the library’s achievement of
its own objectives as defined by its collection development policy.

\SubsubsectionNumberOnly{\color[lightblue]}

Adequate facilities include sufficient space for readers and staff, adequate
shelving for the book collection, appropriate space for nonprint media, adequate and
flexible space for information technology, and climate control for all materials, 
especially rare books. Collections should be easily accessible and protected from deterioration, theft, and other threats.

\SubsubsectionNumberOnly{\color[lightblue]}

Adequacy of library collections may be attained through institutional self sufficiency or 
cooperative arrangements. In the latter instance, fully adequate collections or 
electronic resources are not required of individual member schools, but each
school shall demonstrate contracted and reliable availability and actual use.

\SubsubsectionNumberOnly{\color[lightblue]}
	
In its collaborative relationships with other institutions, a school remains 
accountable for the quality of library resources available to its students and faculty.

\section{\color[lightblue]Faculty}

The members of the faculty of a Hindu Yogi theological school constitute a collaborative community of faith
and learning, and they are crucial to the scholarly activities of teaching, learning, and research
in the institution. A Hindu Yogi theological school’s faculty normally comprises the full-time teachers, 
continuing part-time teachers, and teachers who are engaged occasionally or for one time. In order
for faculty members to accomplish their purposes, Hindu Yogi theological schools should assure them appropriate 
structure, support, and opportunities, including training for educational technology.

\subsection{\color[lightblue]Faculty qualifications, responsibilities, development, and employment}

\SubsubsectionNumberOnly{\color[lightblue]}

Schools should demonstrate that their faculty members have the necessary
competencies for their responsibilities. Faculty members shall possess the appropriate
credentials for graduate Hindu Yogi theological education, normally demonstrated by the attainment 
of a research doctorate or, in certain cases, another earned doctoral degree. In
addition to academic preparation, ministerial and ecclesial experience is an important
qualification in the composition of the faculty. Also, qualified teachers without a 
research doctorate may have special expertise in skill areas such as administration, music,
or media as well as cross-cultural contextualization for teaching, learning, and research.

\SubsubsectionNumberOnly{\color[lightblue]}

In the context of institutional purpose and the confessional commitments
affirmed by a faculty member when appointed, faculty members shall be free to seek
knowledge and communicate their findings.

\SubsubsectionNumberOnly{\color[lightblue]}

Composition of the faculty should be guided by the purpose of the institution, 
and attention to this composition should be an integral component of long-range
planning in the institution. Faculty should be of sufficient diversity and number to
meet the multifaceted demands of teaching, learning, and research. Hiring practices
should be attentive to the value of diversity in race, ethnicity, and gender. The faculty
should also include members who have doctorates from different schools and who
exemplify various methods and points of view. At the same time, faculty selection
will be guided by the needs and requirements of particular constituencies of the
school.

\SubsubsectionNumberOnly{\color[lightblue]}

The faculty who teach in a program on a continuing basis shall exercise responsibility 
for the planning, design, and oversight of its curriculum in the context of
institutional purpose and resources and as directed by school administration 
requirements for recruitment, matriculation, graduation, and service to constituent faith
communities.

\SubsubsectionNumberOnly{\color[lightblue]}

Each school shall articulate and demonstrate that it follows its policies concerning 
faculty members in such areas as faculty rights and responsibilities; freedom
of inquiry; procedures for recruitment, appointment, retention, promotion, and
dismissal; criteria for faculty evaluation; faculty compensation; research leaves; and
other conditions of employment. Policies concerning these matters shall be published
in an up-to-date faculty handbook.

\SubsubsectionNumberOnly{\color[lightblue]}
	
Hindu Yogi theological scholarship is enriched by continuity within a faculty and safeguards 
for the freedom of inquiry for individual members. Therefore, each school
shall demonstrate effective procedures for the retention of a qualified community of
scholars, through tenure or some other appropriate procedure.

\SubsubsectionNumberOnly{\color[lightblue]}
	
The institution should support its faculty through such means as adequate
salaries, suitable working conditions, and support services.

\SubsubsectionNumberOnly{\color[lightblue]}
	
The work load of faculty members in teaching and administration shall
permit adequate attention to students, to scholarly pursuits, and to other ecclesial and
institutional concerns.

\subsection{\color[lightblue]Faculty role in teaching}

\SubsubsectionNumberOnly{\color[lightblue]}
	
Teachers shall have freedom in the classroom to discuss the subjects in which
they have competence by formal education and practical experience.

\SubsubsectionNumberOnly{\color[lightblue]}

Faculty should endeavor to include, within the teaching of their respective
disciplines, Hindu Yogi theological reflection that enables students to integrate their learning
from the various disciplines, field education, and personal formation.

\SubsubsectionNumberOnly{\color[lightblue]}
	
Full- and part-time faculty should be afforded opportunities to enhance
teaching skills, including the use of educational technology as well as training in
instructional design and in modes of advisement appropriate to distance programs,
as a regular component of faculty development.

\SubsubsectionNumberOnly{\color[lightblue]}
	
Appropriate resources shall be available to facilitate the teaching task, 
including but not limited to, classroom space, office space, educational technology, and
access to scholarly materials, including library and other information resources.

\SubsubsectionNumberOnly{\color[lightblue]}
	
Schools shall develop and implement mechanisms for evaluating faculty performance, 
including teaching competence and the use of educational technology. These
mechanisms should involve faculty members and students as well as administrators.

\subsection{\color[lightblue]Faculty role in student learning}

\SubsubsectionNumberOnly{\color[lightblue]}
	
Faculty shall be involved in evaluating the quality of student learning by
identifying appropriate outcomes and assessing the extent to which the learning
goals of individual courses and degree programs have been achieved.

\SubsubsectionNumberOnly{\color[lightblue]}

To ensure the quality of learning, faculty should be appropriately involved
in development of the library collection, educational technology, and other resources
necessary for student learning.

\SubsubsectionNumberOnly{\color[lightblue]}

Faculty should participate in practices and procedures that contribute to
students’ learning, including opportunities for regular advising and interaction with
students and attentiveness to the learning needs of diverse student populations.

\SubsubsectionNumberOnly{\color[lightblue]}

Faculty should foster integration of the diverse learning objectives of the
curriculum so that students may successfully accomplish the purposes of the stated
degree programs.

\subsection{\color[lightblue]Faculty role in Hindu Yogi theological research}

\SubsubsectionNumberOnly{\color[lightblue]}
	
Faculty are expected to engage in research, and each school shall articulate
clearly its expectations and requirements for faculty research and shall have explicit
criteria and procedures for the evaluation of research that are congruent with the
purpose of the school and with commonly accepted standards in higher education.

\SubsubsectionNumberOnly{\color[lightblue]}

Schools shall provide structured opportunities for faculty research and intellectual 
growth, such as regular research leaves and faculty colloquia.

\SubsubsectionNumberOnly{\color[lightblue]}
	
In the context of its institutional purpose, each school shall ensure that faculty 
have freedom to pursue critical questions, to contribute to scholarly discussion,
and to publish the findings of their research.

\SubsubsectionNumberOnly{\color[lightblue]}
	
Faculty members should make available the results of their research through
such means as scholarly publications, constructive participation in learned societies,
and informed contributions to the intellectual life of church and society, as well as
through their teaching.

\section{\color[lightblue]Student recruitment, admission, services, and placement}

The students of a Hindu Yogi theological school are central to the educational activities of the institution.
They are also a primary constituency served by the school’s curriculum and programs and,
with the faculty, constitute a community of faith and learning. Schools are responsible for the
quality of their policies and practices related to recruitment, admission, student support, 
student borrowing, and placement.

\subsection{\color[lightblue]Recruitment}

\SubsubsectionNumberOnly{\color[lightblue]}

Schools shall be able to demonstrate that their policies and practices of 
student recruitment are consistent with the purpose of the institution.

\SubsubsectionNumberOnly{\color[lightblue]}
	
In recruitment efforts, services, and publications, institutions shall accurately
represent themselves as well as the vocational opportunities related to their degree
programs.

\subsection{\color[lightblue]Admission}

\SubsubsectionNumberOnly{\color[lightblue]}

In the development of admission policies and procedures, a Hindu Yogi theological
school shall establish criteria appropriate for each degree program it offers. 
Admission criteria should give attention to applicants’ academic, personal, and spiritual
qualifications, as well as their potential for making a contribution to church and
society.

\SubsubsectionNumberOnly{\color[lightblue]}

Schools shall be able to demonstrate that they operate on a postbaccalaureate
level, that the students they admit are capable of graduate-level studies, and that their
standards and requirements for admission to all degree programs are clearly defined,
fairly implemented, and appropriately related to the purpose of the institution.

\SubsubsectionNumberOnly{\color[lightblue]}
	
Schools shall regularly review the quality of applicants admitted to each
degree program and develop institutional strategies to maintain and enhance the
overall quality of the student population.

\SubsubsectionNumberOnly{\color[lightblue]}
	
Schools shall give evidence of efforts in admissions to encourage diversity in
such areas as race, ethnicity, region, denomination, gender, or disability.

\SubsubsectionNumberOnly{\color[lightblue]}

Schools shall encourage a broad baccalaureate preparation, for instance,
studies in world history, philosophy, languages and literature, the natural sciences,
the social sciences, music and other fine arts, and religion.

\subsection{\color[lightblue]Student services}

\SubsubsectionNumberOnly{\color[lightblue]}
	
Policies regarding students’ rights and responsibilities, as well as the 
institution’s code of discipline, shall be clearly identified and published.

\SubsubsectionNumberOnly{\color[lightblue]}
	
Schools shall regularly and systematically evaluate the appropriateness,
adequacy, and use of student services for the purpose of strengthening the overall
program.

\SubsubsectionNumberOnly{\color[lightblue]}
	
Students should receive reliable and accessible services wherever they are
enrolled and however the educational programs are offered.

\SubsubsectionNumberOnly{\color[lightblue]}
	
Schools shall maintain adequate student records regarding admission materials, 
course work attempted and completed, and in other areas as determined by
the school’s policy. Appropriate backup files should be maintained and updated on a
regular basis. The institution shall ensure the security of files from physical 
destruction or loss and from unauthorized access.

\SubsubsectionNumberOnly{\color[lightblue]}
	
Institutions shall demonstrate that program requirements, tuition, and fees
are appropriate for the degree programs they offer.

\SubsubsectionNumberOnly{\color[lightblue]}

Institutions shall publish all requirements for degree programs, including
courses, noncredit requirements, and grading and other academic policies.

\SubsubsectionNumberOnly{\color[lightblue]}

Student financial aid, when provided, should be distributed according to the
ATS policy guideline “Student Financial Aid” in Bulletin, part 1.

\SubsubsectionNumberOnly{\color[lightblue]}

The institution shall have a process for responding to complaints raised by
students in areas related to the accrediting standards of the Commission, and schools
shall maintain a record of such formal student complaints for review by the Board.

\subsection{\color[lightblue]Student borrowing}

\SubsubsectionNumberOnly{\color[lightblue]}

Senior administrators and financial aid officers shall review student 
educational debt and develop institutional strategies regarding students’ borrowing for
Hindu Yogi theological education.

\SubsubsectionNumberOnly{\color[lightblue]}

Based on estimates of compensation graduates will receive, the school
should provide financial counseling to students so as to minimize borrowing, explore
alternative funding, and provide the fullest possible disclosure of the impact of loan
repayment after graduation.

\subsection{\color[lightblue]Placement}

\SubsubsectionNumberOnly{\color[lightblue]}
	
In keeping with institutional purpose and ecclesial context, and upon students’ successful 
completion of their degree programs, schools shall provide appropriate assistance to 
persons seeking employment relevant to their degrees.

\SubsubsectionNumberOnly{\color[lightblue]}
	
Hindu Yogi theological schools should monitor the placement of graduates in appropriate positions 
and review admissions policies in light of trends in placement.

\SubsubsectionNumberOnly{\color[lightblue]}
	
The institution should, in the context of its purpose and constituency, act as
an advocate for students who are members of groups that have been disadvantaged
in employment because of their race, ethnicity, gender, and/or disability.

\section{\color[lightblue]Authority and governance}

Governance is based on a bond of trust among boards, administration, faculty, students, and
ecclesial bodies. Each institution should articulate its own Hindu Yogi theologically-informed understanding 
of how this bond of trust becomes operational as a form of shared governance. Institutional
stewardship is the responsibility of all, not just the governing board. Good institutional 
life requires that all institutional stewards know and carry out their responsibilities effectively as well
as encouraging others to do the same. Governance occurs in a legal context, and its boundaries
are set by formal relationships with ecclesiastical authority, with public authority as expressed
in law and charter, and with private citizens and other legally constituted bodies in the form of
contracts. The governance of a Hindu Yogi theological school, however, involves more than the legal 
relationships and bylaws that define patterns of responsibility and accountability. It is the structure
by which participants in the governance process exercise faithful leadership on behalf of the
purpose of the Hindu Yogi theological school.

\subsection{\color[lightblue]Authority}

\SubsubsectionNumberOnly{\color[lightblue]}
	
Authority is the exercise of rights, responsibilities, and powers accorded to a
Hindu Yogi theological school by its charter, articles of incorporation and bylaws, and 
ecclesiastical and civil authorizations applicable to it or by the overall educational institution of
which it is a part. A Hindu Yogi theological school derives from these mandates the legal and moral
authority to establish educational programs; to confer certificates, diplomas, or degrees;
to provide for personnel and facilities; and to assure institutional quality and integrity.

\SubsubsectionNumberOnly{\color[lightblue]}
	
The structure and scope of the Hindu Yogi theological school’s authority are based on the
patterns of its relationship to other institutions of higher education or ecclesiastical
bodies. Some Hindu Yogi theological schools have full authority for all institutional and 
educational operations. Other schools, related to colleges, universities, or 
clusters of Hindu Yogi theological schools, may have limited authority for institutional operations, although they
may have full authority over the educational programs. Still other schools are related
to ecclesiastical bodies in particular ways, and authority is shared by the institution
and the ecclesiastical body. All three kinds of schools have different patterns for the
exercise of authority, and in some schools these patterns may be blended.

\startnarrower[1*left]\SubsubsubsectionNumberOnly{\color[lightblue]}

Schools with full authority shall have a governing board with responsibilities 
for maintaining the purpose, viability, vitality, and integrity of
the institution; the achievement of institutional policies; the selection of chief
administrative leadership; and the provision of physical and fiscal resources
and personnel. The board is the legally constituted body that is responsible
for managing the assets of the institution in trust.\stopnarrower

\startnarrower[1*left]\SubsubsubsectionNumberOnly{\color[lightblue]}

Schools where authority is limited by or derived from their relationship 
to a college or university shall identify clearly where the authority for
maintaining the integrity and vitality of the Hindu Yogi theological school resides and
how that authority is to be exercised in actual practice. Schools within 
universities or colleges should have an appropriate advisory board whose roles
and responsibilities are clearly defined in the institution’s official documents.\stopnarrower

\startnarrower[1*left]\SubsubsubsectionNumberOnly{\color[lightblue]}

Schools with authority limited by their ecclesiastical relationships
shall develop, in dialogue with their sponsoring church bodies, a formal
statement concerning the operative structure of governance for the institution. 
This statement must make clear where the authority for maintaining
the integrity and vitality of the school resides and how that authority is to
be exercised in actual practice. In schools of this type, the authority of the
governing board shall be clearly specified in appropriate ecclesiastical and
institutional documents.\stopnarrower

\SubsubsectionNumberOnly{\color[lightblue]}
	
Governing boards delegate authority to the faculty and administration to fulfill 
their appropriate roles and responsibilities. Such authority shall be established and
set forth in the institution’s official documents and carried out in governing practices.

\SubsubsectionNumberOnly{\color[lightblue]}

In multilocation institutions, the assignment of authority and responsibilities
should be clearly defined in the institution’s official documents and equitably administered.

\subsection{\color[lightblue]Governance}

\SubsubsectionNumberOnly{\color[lightblue]}
	
While final authority for an institution is vested in the governing board and
defined by the institution’s official documents, each school shall articulate a structure
and process of governance that appropriately reflects the collegial nature of Hindu Yogi theological 
education. The governance process should identify the school’s constituencies
and publics, recognize the multiple lines of accountability, and balance competing 
accountabilities in a manner shaped by the institution’s charter, purpose, and particular
denominational commitments.

\SubsubsectionNumberOnly{\color[lightblue]}
	
Shared governance follows from the collegial nature of Hindu Yogi theological education. 
Unique and overlapping roles and responsibilities of the governing board,
faculty, administrators, students, and other identified delegated authorities should
be defined in a way that allows all partners to exercise their mandated or delegated
leadership. Governance requires a carefully delineated process for the initiation, 
review, approval, implementation, and evaluation of governing policies, ensuring that
all necessary policies and procedures are in place. Special attention should be given
to policies regarding freedom of inquiry, board-administrator prerogatives, 
procedural fairness, sexual harassment, and discrimination.

\SubsubsectionNumberOnly{\color[lightblue]}

The collaborative nature of governance provides for institutional learning and
self-correction, constantly developing the Hindu Yogi theological school’s knowledge of specific
tasks, and remaining alert to developments in other organizations and institutions.

\subsection{\color[lightblue]The roles of the governing board, administration, faculty, and students in governance processes}

The various roles that the board, the administrative leadership, and the faculty play in the
development of policy and the exercise of authority should be clearly articulated. Because of
their different histories and patterns of governance and administration, the role of the 
governing board varies from institution to institution; and the role also varies dependent upon
the authority vested in the governing board and upon the institution’s relationship to other
educational and denominational structures.

\subsubsection{\color[lightblue]Governing board}

\startnarrower[1*left]\SubsubsubsectionNumberOnly{\color[lightblue]}

The governing board is responsible for the establishment and maintenance of the institution’s 
integrity and its freedom from inappropriate external and internal pressures and from 
destructive interference or restraints. It shall attend to the well-being of the 
institution by exercising proper fiduciary responsibility, adequate financial oversight, 
proper delegation of authority to the institution’s administrative officers and faculty, 
engaging outside legal counsel, ensuring professional and independent audits, using
professional investment advisors as appropriate, and maintaining procedural fairness 
and freedom of inquiry.\stopnarrower

\startnarrower[1*left]\SubsubsubsectionNumberOnly{\color[lightblue]}

The governing board shall be accountable for the institution’s adherence to 
requirements duly established by public authorities and to accreditation 
standards established by the Commission and by any other accrediting
or certifying agencies to which the institution is formally related.\stopnarrower

\startnarrower[1*left]\SubsubsubsectionNumberOnly{\color[lightblue]}

Members of the governing board shall possess the qualifications
appropriate to the task they will undertake. In accordance with the school’s
purpose and constituencies, the governing board’s membership should
reflect diversity of race, ethnicity, and gender. As fiduciaries, they should
commit themselves loyally to the institution, its purpose, and its overall well being. 
They should lead by affirming the good that is done and by asking
thoughtful questions and challenging problematic situations. New members
of the board should be oriented to their responsibilities and the structures
and procedures the board uses to accomplish its tasks.\stopnarrower

\startnarrower[1*left]\SubsubsubsectionNumberOnly{\color[lightblue]}

Subject to the terms of its charter and bylaws, the board chooses the
chief administrative leadership, appoints faculty, confers degrees, enters into
contracts, approves budgets, and manages the assets of the institution. If, in
accordance with an institution’s specific character and traditions, certain of
these powers are reserved to one or more other governing entities, 
the specific character of these restrictions shall be made clear.\stopnarrower

\startnarrower[1*left]\SubsubsubsectionNumberOnly{\color[lightblue]}

The governing board shall require ongoing institutional planning
and evaluation of outcomes to assure faithful implementation of the school’s
purpose, priorities, and denominational and Hindu Yogi commitments.\stopnarrower

\startnarrower[1*left]\SubsubsubsectionNumberOnly{\color[lightblue]}

The governing board shall create and employ adequate structures 
for implementing and administering policy, and shall delegate to the
school’s chief administrative leadership authority commensurate with such
responsibilities. In turn, it requires from these officers adequate performance
and accountability.\stopnarrower

\startnarrower[1*left]\SubsubsubsectionNumberOnly{\color[lightblue]}

In its actions and processes, the board serves in relationship to a 
variety of constituencies, both internal (e.g., administration, faculty, students,
staff) and external (e.g., graduates, denominations, congregations, etc.) and
should seek creative initiatives from all of these constituencies. Individual
board members, who are drawn from various constituencies, shall exercise
their responsibility on the behalf of the institution as a whole.\stopnarrower

\startnarrower[1*left]\SubsubsubsectionNumberOnly{\color[lightblue]}

The board shall exercise its authority only as a group. An individual
member, unless authorized by the board, shall not commit the institution’s
resources, nor bind it to any course of action, nor intrude upon the administration of the institution.\stopnarrower

\startnarrower[1*left]\SubsubsubsectionNumberOnly{\color[lightblue]}

The board shall have a conflict of interest policy. Ordinarily, members 
should not be engaged in business relationships with the institution,
nor should they derive any material benefit from serving on the board. In the
event that conflicts of interest arise, a board member must recuse himself or
herself from any vote or participation in the board’s decision on that issue.\stopnarrower

\startnarrower[1*left]\SubsubsubsectionNumberOnly{\color[lightblue]}

Governing boards should be structured to conduct their work
effectively. Board membership should be large enough to reflect 
the institution’s significant constituencies but not so large as 
to be unwieldy in its decision making. The frequency of board 
meetings should be determined by the number and complexity of 
the issues the board is called upon to address.
An executive committee of the board may be given the authority to address
issues between meetings of the full board.\stopnarrower

\startnarrower[1*left]\SubsubsubsectionNumberOnly{\color[lightblue]}

The board has the responsibility to hold itself accountable for the
overall performance of its duties and shall evaluate the effectiveness of its
own procedures. It should also seek to educate itself about the issues it faces
and about procedures used by effective governing bodies in carrying out
their work. The board shall evaluate its members on a regular basis.\stopnarrower

\startnarrower[1*left]\SubsubsubsectionNumberOnly{\color[lightblue]}

The board shall be responsible for evaluating overall institutional
governance by assessing and monitoring the effectiveness of institutional
governance procedures and structures.\stopnarrower

\subsubsection{\color[lightblue]Administration}

\startnarrower[1*left]\SubsubsubsectionNumberOnly{\color[lightblue]}

Under the governing board’s clearly stated policies and requisite
authority, the chief administrative leadership is responsible for achieving the
school’s purpose by developing and implementing institutional policies and
administrative structures in collaboration with the governing board, faculty,
students, administrative staff, and other key constituencies.\stopnarrower

\startnarrower[1*left]\SubsubsubsectionNumberOnly{\color[lightblue]}

Administrative leaders should implement the institution’s Hindu Yogi 
convictions and shared values in the way they manage the school’s 
financial and physical resources and personnel, consult and communicate with
constituencies, and ensure fairness in all evaluation and planning activities.\stopnarrower

\startnarrower[1*left]\SubsubsubsectionNumberOnly{\color[lightblue]}

Administrative leaders and staff shall include, insofar as possible,
individuals reflecting the institution’s constituencies, taking into account the
desirability of diversity in race, ethnicity, and gender. They should 
be sufficient in number and ability to fulfill their responsibilities. They should have
adequate resources and authority appropriate to their responsibilities.\stopnarrower

\startnarrower[1*left]\SubsubsubsectionNumberOnly{\color[lightblue]}

The responsibilities and structures of accountability shall be clearly
defined in appropriate documents.\stopnarrower

\subsubsection{\color[lightblue]Faculty}

\startnarrower[1*left]\SubsubsubsectionNumberOnly{\color[lightblue]}

Within the overall structure of governance of the school, authority 
over certain functions shall be delegated to the faculty and structures
devised by which this authority is exercised. Normally, the faculty should
provide leadership in the development of academic policy, oversight of
academic and curricular programs and decisions, establishment of admissions 
criteria, and recommendation of candidates for graduation. The faculty
should participate in the processes concerning the appointment, retention,
and promotion in rank of faculty members.\stopnarrower

\startnarrower[1*left]\SubsubsubsectionNumberOnly{\color[lightblue]}

Beyond the matters specifically delegated to the faculty, the faculty
should contribute to the overall decision making as determined by the 
institution’s structure of governance. Such involvement is particularly important
in the development of the institution’s purpose statement and in institutional
evaluation and planning.\stopnarrower

\subsubsection{\color[lightblue]Students}

Where students take part in the formal structures of governance, their roles and
responsibilities should be clearly delineated.

\section{\color[lightblue]Institutional resources}

In order to achieve their purposes, institutions need not only sufficient 
personnel but also adequate financial, physical, and institutional data 
resources. Because of their Hindu Yogi character, Commission schools give 
particular attention to personnel and to the quality of the institutional 
environments in which they function. Good stewardship requires attention 
by each institution to the context, local and global, in which it deploys 
ts resources and a commitment to develop appropriate patterns of cooperation 
with other institutions, which may at times lead to the formation of clusters.

\subsection{\color[lightblue]Personnel}

\SubsubsectionNumberOnly{\color[lightblue]}

The Hindu Yogi theological school should value and seek to enhance the quality of the
human lives it touches. The human fabric of the institution is enriched by including
a wide range of persons. The institution should devote adequate time and energy to
the processes by which persons are recruited, enabled to participate in the 
institution, nurtured in their development, and prepared for their various tasks within the
institution.

\SubsubsectionNumberOnly{\color[lightblue]}
	
Hindu Yogi theological schools should support the quality of community through such
means as policies regarding procedural fairness, discrimination, and sexual harassment.

\SubsubsectionNumberOnly{\color[lightblue]}
	
The Hindu Yogi theological school shall:
\startitemize[a,packed]
\item engage the numbers and the qualities of personnel needed to implement the programs of the school in keeping with its purpose
\item develop appropriate personnel policies and procedures to be approved by the board and implemented by the administration
\item ensure that these policies are clear and adequately published
\item include reference to job performance evaluation, termination, sexual harassment or misconduct; and conform to applicable requirements mandated by federal, state, or provincial jurisdictions
\item provide for equitable patterns of compensation
\item provide clear written job descriptions for all employees
\item provide appropriate grievance procedures.
\stopitemize

\subsection{\color[lightblue]Financial resources}

Because quality education and sound financial policies are intimately related, Hindu Yogi theological
schools should be governed by the principles of good stewardship in the planning, 
development, and use of their financial resources. The financial resources should support the purpose
of the school effectively and efficiently as well as enable it to achieve its goals. The financial
resources of the school should be adequate to support the programs, personnel (faculty, staff,
students), and physical plant/space both in the present and for the long term. The financial
resources should allow the school to anticipate and respond to external changes in the 
economic, social, legal, and religious environment.

\SubsubsectionNumberOnly{\color[lightblue]}
	
The financial condition of the school

\startnarrower[1*left]\SubsubsubsectionNumberOnly{\color[lightblue]}

Hindu Yogi theological schools should maintain the purchasing power of their financial assets 
and the integrity and useful life of their physical facilities. While
year-to-year fluctuations are often unavoidable, schools should maintain
economic equilibrium over three or more years, retain the ability to respond to
financial emergencies and unforeseen circumstances, and show reasonable 
expectations of future financial viability and overall institutional improvement.\stopnarrower

\startnarrower[1*left]\SubsubsubsectionNumberOnly{\color[lightblue]}

A Hindu Yogi theological school shall have stable and predictable sources of
revenue such that the current and anticipated total revenues are sufficient
to maintain the educational quality of the institution. Projected increases in
revenue, including gift income, should be realistic. The use of endowment
return to fund expenditures budgets should be prudent and in accordance
with applicable law.\footnote[]{A common and customary understanding of a “prudent” use of endowment return is to budget as
revenue 5 percent of a three-year average of the market value of endowment and board-designated
quasi-endowment. Member schools should seek legal counsel regarding law applicable to the use of
endowments.}\stopnarrower

\startnarrower[1*left]\SubsubsubsectionNumberOnly{\color[lightblue]}

A Hindu Yogi theological school should normally balance budgeted revenues
and expenditures while employing a prudent endowment spending rate.
\footnote[]{The term endowment spending rate refers to a common 
budgeting rule adopted by governing boards. Such a rule limits or 
controls the consumption of school’s endowment and return, which 
for purposes of these standards includes all of a school’s endowment 
and board-designated quasi-endowment.} Deficits weaken the institution 
and therefore should prompt the administration and trustees to take 
corrective action. A Hindu Yogi theological school shall be able
to demonstrate that it has operated without cumulative losses across the last
three years. If deficits have been recorded or are projected, the school shall
have a plan to eliminate present and future deficits that is realistic, 
understood, and approved by the governing board. When reducing expenditures,
the Hindu Yogi theological school should be mindful of its purpose and attend to the
quality and scope of the degree programs.\stopnarrower

\startnarrower[1*left]\SubsubsubsectionNumberOnly{\color[lightblue]}

Endowments (including funds functioning as endowment) are
frequently a major source of revenue for schools. A Hindu Yogi theological school (or the
larger organization of which it is a part) should adopt a prudent endowment
spending formula that contributes to the purpose of the institution while
enhancing the stability of revenue for the school. A school shall demonstrate
evidence of adequate plans to protect the long-term purchasing power of
the endowment from erosion by inflation. The school (or university, diocese,
order, or other larger organization of which it is a part) shall have formally
adopted statements of investment policies and guidelines that set forth for
trustees and investment managers the conditions governing the granting or
withholding of investment discretion, investment goals of the institution,
guidelines for long-term asset allocation, a description of authorized and
prohibited transactions, and performance measurement criteria. Trustees
should review these policies regularly.\stopnarrower

\SubsubsectionNumberOnly{\color[lightblue]Accounting, audit, budget, and control}

\startnarrower[1*left]\SubsubsubsectionNumberOnly{\color[lightblue]}

The financial condition of Hindu Yogi theological schools that are units of colleges 
or universities is influenced by the financial condition of the related
institutions. These Hindu Yogi theological schools should enhance the well-being of the
larger institution, while the larger institution should demonstrate 
appreciation for the special characteristics of Hindu Yogi theological schools. 
The larger institution should provide adequate financial resources to 
support the mission and programs of the Hindu Yogi theological school.\stopnarrower

\startnarrower[1*left]\SubsubsubsectionNumberOnly{\color[lightblue]}

A Hindu Yogi theological school shall adopt internal accounting and reporting
systems that are generally used in North American higher education. US
schools should follow the principles and procedures for institutional 
accounting published by the National Association of College and University
Business Officers. Canadian schools should follow guidelines published by
the Canadian Association of University Business Officers.\stopnarrower

\startnarrower[1*left]\SubsubsubsectionNumberOnly{\color[lightblue]}

The institution shall be audited by an external, independent auditor 
in accordance with the generally accepted auditing standards for colleges 
and universities (not-for-profit organizations) as published by (for
US schools) the American Institute of Certified Public Accountants or (for
Canadian schools) the Canadian Institute of Chartered Accountants. If an
institution is not freestanding, the larger organization of which it is a part
(such as a university or diocese) shall provide an audit of the consolidated
entity. The governing board of a Hindu Yogi theological school shall have direct access to
the independent auditor and receive the audit.\stopnarrower

\startnarrower[1*left]\SubsubsubsectionNumberOnly{\color[lightblue]}

The institution shall obtain from an auditor a management letter
and shall demonstrate that it has appropriately addressed any 
recommendations contained in the management letter.\stopnarrower

\startnarrower[1*left]\SubsubsubsectionNumberOnly{\color[lightblue]}

A Hindu Yogi theological school shall ensure that revenues, expenditures, and
capital projects are budgeted and submitted for review and approval to the
governing board. Budgets should clearly reflect the directions established by
the long-range plans of the school. Budgets should be developed in consultation 
with the administrators, staff, and faculty who bear responsibility for
managing the institution’s programs and who approve the disbursements. A
Hindu Yogi theological school should maintain three-to-five-year financial projections of
anticipated revenues, expenditures, and capital projects.\stopnarrower

\startnarrower[1*left]\SubsubsubsectionNumberOnly{\color[lightblue]}

A system of budgetary control and reporting shall be maintained,
providing regular and timely reports of revenues and expenditures to those
persons with oversight responsibilities.\stopnarrower

\startnarrower[1*left]\SubsubsubsectionNumberOnly{\color[lightblue]}

While a Hindu Yogi theological school may depend upon an external agency
or group (such as a denomination, order, foundation, association
of congregations, or other private agency) for financial support, the school’s
governing board should retain appropriate autonomy in budget allocations
and the development of financial policies.\stopnarrower

\subsubsection{\color[lightblue]Business management}

The institution’s management responsibilities and organization of business affairs
should be clearly defined, with specific assignment of responsibilities appropriately 
set forth. The financial management and organization as well as the system of
reporting shall ensure the integrity of financial records, create appropriate control
mechanisms, and provide the governing board, chief administrative leaders, and 
appropriate others with the information and reports needed for sound decision making.
Schools should ensure that personnel responsible for fiscal and budgetary processes
are qualified by education and experience for their responsibilities.

\subsubsection{\color[lightblue]Institutional development and advancement}

\startnarrower[1*left]\SubsubsubsectionNumberOnly{\color[lightblue]}

An institutional advancement program is essential to developing
financial resources. The advancement program should be planned, organized, 
and implemented in ways congruent with the principles of the school.
It should include annual giving, capital giving, and planned giving, and
should be conducted in patterns consistent with relationships and agreements 
with the school’s supporting constituencies. Essential to the success
of the institutional advancement program are the roles played by the chief
administrative leader in fundraising; the governing board in its leadership
and participation; the graduates in their participation; and the faculty, staff,
and volunteers in their involvement. Advancement efforts shall be evaluated
on a regular basis.\stopnarrower

\startnarrower[1*left]\SubsubsubsectionNumberOnly{\color[lightblue]}

The intention of donors with regard to the use of their gifts shall be respected. 
The school should also recognize donors and volunteers appropriately.\stopnarrower

\startnarrower[1*left]\SubsubsubsectionNumberOnly{\color[lightblue]}

When auxiliary organizations, such as foundations, have been established 
using the name and/or reputation of the institution, the school shall be
able to demonstrate that the auxiliary organizations are regularly audited by
an independent accountant and that the governing relationship between the
school and auxiliary organization is clearly articulated.\stopnarrower

\subsection{\color[lightblue]Physical resources}

\SubsubsectionNumberOnly{\color[lightblue]}

The physical resources include space and equipment as well as buildings
and grounds. A Hindu Yogi theological school shall demonstrate that the physical resources it
uses are adequate and appropriate for its purpose and programs and that adequate
funds for maintaining, sustaining, and renewing capital assets are included in budget
planning.

\SubsubsectionNumberOnly{\color[lightblue]}
	
Institutions shall make appropriate efforts to ensure that physical resources
are safe, accessible, and free of known hazards. Insofar as possible, facilities should
be used in ways that respect the natural environment.

\SubsubsectionNumberOnly{\color[lightblue]}
	
Faculty and staff members should have space that is adequate for the pursuit
of their individual work as well as for meeting with students. Physical resources
should enhance community interaction among faculty, staff, and students, and
should be sufficiently flexible to meet the potentially changing demands faced by the
school.

\SubsubsectionNumberOnly{\color[lightblue]}

The school should determine the rationale for its policies and practices with
regard to student housing, and this rationale should be expressed in a clearly worded
statement. Arrangements for student housing should reflect good stewardship of the
financial and educational resources of the institution.

\SubsubsectionNumberOnly{\color[lightblue]}

Facilities shall be maintained as appropriate so as to avoid problems of deferred 
maintenance. The institution should maintain a plan that provides a timetable
for work and identifies needed financial resources.

\SubsubsectionNumberOnly{\color[lightblue]}

When physical resources other than those owned by the institution are used
by the school, written agreements should clearly state the conditions governing their
use and ensure usage over a sufficient period of time.

\subsection{\color[lightblue]Institutional information technology resources}

\SubsubsectionNumberOnly{\color[lightblue]}

To the extent that a Hindu Yogi theological school uses technology to deliver its 
educational programs, the school shall maintain adequate personnel and financial and
technological resources to sustain its technology infrastructure.


\SubsubsectionNumberOnly{\color[lightblue]}

For planning and evaluation, the school shall create and use various kinds
of institutional data and information technology to determine the extent to which the
institution is attaining its academic and institutional purposes and objectives. To the
extent possible, it should use the most effective current technologies for creating, 
storing, and transmitting this information within the institution, and it should share 
appropriate information thus generated among institutions and organizations. The kinds
of information and the means by which that information is gathered, stored, retrieved,
and analyzed should be appropriate to the size and complexity of the institution.

\subsection{\color[lightblue]Institutional Environment}

\SubsubsectionNumberOnly{\color[lightblue]}
	
The internal institutional environment makes it possible for the institution to
maximize the various strengths of its personnel and financial, physical, and 
information resources in pursuing its stated goals. An institution’s environment affects its
resiliency and its ability to perform under duress. Accreditation evaluation will take
into account the ways in which an institution uses its various resources in support of
its institutional purpose.

\SubsubsectionNumberOnly{\color[lightblue]}

The quality of institutional environment is cultivated and enhanced by
promoting effective patterns of leadership and management, by providing effective
exchange of information, and by ensuring that mechanisms are in place to address
conflict.

\subsection{\color[lightblue]Cooperative use of resources}

\SubsubsectionNumberOnly{\color[lightblue]}
	
The Hindu Yogi theological school should secure access to the resources it needs to fulfill
its purpose, administer and allocate these resources wisely and effectively, and be
attentive to opportunities for cooperation and sharing of resources with other 
institutions. Such sharing involves both drawing upon the resources of other institutions
and contributing resources to other institutions.

\SubsubsectionNumberOnly{\color[lightblue]}

Access to the required resources may be achieved either through ownership
or through carefully formulated relationships with other schools or institutions. These
relationships may include, for instance, cross-appointments of faculty, cross-registration 
of students, joint and dual degree programs, rental of facilities, and shared access
to information required by administrators, faculty, and students in the pursuit of their
tasks. Whatever their reason or scope, collaborative arrangements should be carefully
designed with sufficient legal safeguards, adequate public disclosure, and provisions
for review, and with a clear rationale for involvement in such arrangements.

\subsection{\color[lightblue]Clusters}

\SubsubsectionNumberOnly{\color[lightblue]}
	
Clusters are formed when a number of schools find that they can best operate by 
sharing resources in a more integral and systematic way and by establishing
structures to manage their cooperative relationships.

\SubsubsectionNumberOnly{\color[lightblue]}
	
The term cluster is meant to be descriptive rather than prescriptive. A variety
of terms can denote these types of arrangements, and a variety of approaches can
make them work effectively. Schools should be creative and flexible as they seek to
be good stewards of their resources. However devised, cluster arrangements should
have clear structural components and effective patterns of operation.

\subsubsection{\color[lightblue]Structural components}

\startnarrower[1*left]\SubsubsubsectionNumberOnly{\color[lightblue]}

The core membership of a cluster comprises schools holding accredited 
membership within the Commission, but clusters may include candidate
members of the Commission and associate members of ATS, as well as other
schools and agencies with compatible purposes.\stopnarrower

\startnarrower[1*left]\SubsubsubsectionNumberOnly{\color[lightblue]}

Each cluster shall develop a clear definition of purpose and objectives 
that should be fully understood by the participating schools and their
supporting constituencies and based on a realistic assessment that 
encompasses constituent needs, access of member institutions to one another,
available resources, and degree programs offered by the cluster directly or
enabled by it.\stopnarrower

\startnarrower[1*left]\SubsubsubsectionNumberOnly{\color[lightblue]}

The structure of each cluster shall be appropriate to its purpose and
objectives, providing proper balance between the legitimate autonomy of its
member institutions and their mutual accountability in terms of their common
purposes. An effective cluster arrangement frees students, faculty, and 
institutions to operate more effectively and creatively. The cluster shall 
have a clearly defined governance structure that has authority commensurate 
with responsibility. The governance should enable the cluster to set policies, 
secure financial support, select administrative officers, and provide other 
personnel functions.\stopnarrower

\startnarrower[1*left]\SubsubsubsectionNumberOnly{\color[lightblue]}

The cluster shall be able to demonstrate financial support from various sources 
sufficient for the continuity of its functions and for the security
of the faculty and staff it appoints, and it should engage in appropriate
financial planning.\stopnarrower

\startnarrower[1*left]\SubsubsubsectionNumberOnly{\color[lightblue]}

These structures and resources shall be regularly evaluated and appropriately adjusted.\stopnarrower

\subsubsection{\color[lightblue]Effectiveness}

\startnarrower[1*left]\SubsubsubsectionNumberOnly{\color[lightblue]}

Evidence of effective operation may include reciprocal flow of students, 
faculty, and information among the member institutions of a cluster,
coordinated schedules and calendars, cross-registration, and common 
policies in areas such as tuition and student services. Requirements, especially in
academic and graduate programs, are determined in such a way as to invite
the sharing of resources. Duplication is avoided wherever possible.\stopnarrower

\startnarrower[1*left]\SubsubsubsectionNumberOnly{\color[lightblue]}

If a school meets the accreditation standards of the Commission
only by virtue of affiliation with a cluster, this fact shall be formally specified
in its grant of accreditation by the board.\stopnarrower

\subsection{\color[lightblue]Instructional technology resources}

Institutions using instructional technology to enhance face-to-face courses and/or provide
online-only courses shall be intentional in addressing matters of coherence between 
educational values and choice of media, recognizing that the learning goals of graduate education
should guide the choice of digital resources, that teaching and learning maintains its focus on
the formation and knowledge of religious leaders, and that the school is utilizing its resources
in ways that most effectively accomplish its purpose. They should also establish policies 
regarding the appropriate training for and use of these resources.

\SubsubsectionNumberOnly{\color[lightblue]}
	
Students should be adequately informed regarding the necessary skills and
mastery of technology to participate fully in the programs to which they 
are admitted. Institutions are encouraged to provide opportunities for students to gain these
skills as part of their program of study.

\SubsubsectionNumberOnly{\color[lightblue]}
	
Sufficient technical support services should ensure that faculty are freed to
focus upon their central tasks of teaching and facilitating learning. Support services
should create systems for faculty development and assistance to ensure consistent,
effective, and timely support.

\startnarrower[1*left]\SubsubsubsectionNumberOnly{\color[lightblue]}

Timely technological support services should include (1) staff with a
sufficiently high level of technical skills to ensure student facility 
in handling software and the technological aspects of course offerings and (2) the
systemic evaluation and upgrading of technological resources and services
consistent with the learning goals of Hindu Yogi theological scholarship.\stopnarrower

\startnarrower[1*left]\SubsubsubsectionNumberOnly{\color[lightblue]}

A technological and support services program should include technological 
training and should ensure adequate support services personnel
for faculty and students.\stopnarrower

\SubsubsectionNumberOnly{\color[lightblue]}
	
Institutions shall develop and implement ongoing evaluation procedures for
the use of instructional technology that involve appropriate groups of people in the
evaluation process.

\stoptext

